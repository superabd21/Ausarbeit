%------------------------------------------------
% Latex-Grundgerüst für Seminarausarbeitungen
%
% zu Erzeugen mit
%   pdflatex ausarbeitung.tex
%
% von Carsten Gutwenger, Nils Kriege, Denis Kurz
% 
%------------------------------------------------

\documentclass[a4paper,12pt,twoside,headsepline]{scrartcl}
\usepackage[utf8]{inputenc}

\usepackage[automark]{scrlayer-scrpage}                     % Kopf-/Fußzeilen
\usepackage[ngerman]{babel}                                 % deutsche Sprache
\usepackage[T1]{fontenc}                                    % Unterstützung für Umlaute mit Fonts

\usepackage[pdftex]{graphicx,color}                         % Einbetten von Grafiken, Farbe
\usepackage{subcaption}                                     % subfigures
\usepackage{url}                                            % \url{}-Kommando
\usepackage[pdftex,pdfpagelabels]{hyperref}                 % Hyperlinks im PDF
\usepackage[round]{natbib}                                  % Literaturverzeichnis
\usepackage{doi}                                            % DOIs

\usepackage[section,boxed]{algorithm}                       % Floating-Umgebung für Algorithmen
\usepackage{algpseudocode}                                  % Pseudocode für Algorithmen
\usepackage{amsmath}                                        % Mathe-Befehle: \frac{m}{n}, \log, \mathcal O, ...
\usepackage{amssymb}                                        % Mathe-Symbole: \mathbb N, ...
\usepackage{amsthm}                                         % Theorem-Umgebungen, siehe \newtheorem und letzte Subsection
\usepackage{tikz}                                           % Zeichnungen, z.B. von Graphen
\usepackage{todonotes}                                      % kaum zu übersehende TODO-Anmerkungen

% Header und Footer
\pagestyle{scrheadings}
\ofoot{}\cfoot{}\ifoot{}
\lehead{\pagemark}
\rehead{\slshape\leftmark}
\lohead{\slshape\leftmark}
\rohead{\pagemark}

\newtheorem{definition}{Definition}
\newtheorem{beispiel}{Beispiel}
\newtheorem{bemerkung}{Bemerkung}

\newtheorem{proposition}{Proposition}
\newtheorem{lemma}{Lemma}
\newtheorem{satz}{Satz}
\newtheorem{theorem}{Theorem}
\newtheorem{korollar}{Korollar}

\begin{document}

%------------------------------------------------
% Titelseite
%------------------------------------------------

\begin{titlepage}
\sffamily
\vspace*{-3cm}
\includegraphics[width=8cm]{tud_logo_rgb}

\vspace*{4cm}
\begin{center}
\large
{\Large Proseminarausarbeitung} \\[1ex]
{\LARGE\textbf{Operationelle Semantik}} \\[3ex]
Abdulrahman Alrefai \\[1ex]
\today \\[7ex]
im Rahmen des Seminars \\[1ex]
{\Large\textbf{Software Verification}} \\[1ex]
von Prof. Dr. Falk Howar \\[1ex]
Sommersemester 2019
\end{center}


\noindent
\textbf{Basierend auf:}\\
Glynn Winskel. The formal semantics of programming languages: an introduction, \textit{MIT press}


\vspace*{\fill}
\noindent
Fakultät für Informatik\\
Lehrstuhl 14 -- Software Engineering\\
TU Dortmund
\end{titlepage}

%------------------------------------------------
% Inhaltsverzeichnis
%------------------------------------------------

\tableofcontents
\clearpage

\todo[inline]{Alle \texttt{todo}-Befehle entfernen}

%------------------------------------------------
% Einführung
%------------------------------------------------

\section{Einführung}

Die Bedeutung der präzisen Semantik für Programmier- und Spezifikationssprachen wurde seit den sechziger Jahren mit der Entwicklung der ersten höheren Programmiersprachen anerkannt. 
Die Verwendung der operativen Semantik, d. h. einer Semantik, die explizit beschreibt, wie Programme schrittweise durchgeführt werden, und die möglichen Zustandstransformationen durchführen. 
Beim Erlernen einer neuen Programmiersprache wird normalerweise nur die Syntax angeschaut, um zu wissen, wie bestimmte Sprachkonstrukte geschrieben werden. Syntax behandelt jedoch nur korrekt gebildete Sätze. 
Dies bedeutet, dass der Programmierer verloren geht, wenn es notwendig wird, zu überprüfen, ob das angegebene Programm tatsächlich den beabsichtigten Vorgang abläuft. Es ist jedem Programmierer passiert, trotz eines syntaktisch korrekten Programm falsche Ergebnisse zurückgeliefert werden. Der Grund dafür ist, dass das Programm nicht unter allen Umständen semantisch korrekt ist. Programme sind normalerweise zu groß und zu komplex, um sie zu verstehen und zu überprüfen, daher ist es wichtig, dass man ein tiefes Verständnis über operationale Semantik hat.
 Die Semantik für eine Programmiersprache stellt abstrakte Entitäten bereit, die nur die relevanten Merkmale aller möglichen Ausführungen darstellen, und ignoriert Details, die für die Korrektheit von Implementierungen nicht relevant sind. Operationale Semantik erzeugt ein gekennzeichnetes Übergangssystem, dessen Zustände die geschlossenen Begriffe über einer algebraischen Signatur sind und deren Übergänge zwischen Zuständen induktiv aus einer Sammlung  Übergangsregeln der Form erhalten werden:
 $\frac{ permises} {Conclusion}$ . (Wenn alle Prämissen stimmen, dann tun die Schlussfolgerungen)

\section{Zustände}
Beschreibung der Bedeutung einer Programmierung
Sprache, indem spezifiziert wird, wie sie auf einem abstrakten Computer ausgeführt wird.
Sie wird im Speicher durchgeführt und die verschiedenen verfügbaren Ressourcen verwenden. 
Der Inhält der speicher ändert sich :
\begin{itemize}
	\item Variablen werden aktualisiert .
	\item neue Datenbank .
\end{itemize}
Außerdem kann der Inhalt einer variablen im Moment des
Programmstarts  einen großen Einfluss auf das Endergebnis haben.
SO müssen wir den Status des Speichers während der Ausführung berücksichtigen.
wir interessieren uns nur einen Teil vom Speicher(Werte der Variablen ).
Jetzt können wir Status so definieren:
 Es ist eine Funktion , die Variablen Werte  gibt.
 Die Menge der Zustände $\Sigma$ besteht aus Funktionen $\sigma$ : Loc$\rightarrow$ N .Somit ist $\sigma$ (X) der Wert oder Inhalt der Stelle X im Zustand $\sigma$ .
 \subsection{Operationelle Semantik}
Bedeutung eines Programms von einem bestimmten Zustand (Startzustand) und Beobachtung, wie Status im Ende der Ausführung des Programms geändert wird (Endzustand).

\section{Operationelle Semantik für Einfache Imperative Sparache}


	\textbf{syntaktische Menge mit IMP}
		\begin{itemize}
			\item n $\in$ Num   (positive und negative ganze Zahlen mit Null)
			\item b $\in$ BExp  (Boolean Expression)
			\item a $\in$ AExp   (Arithmetic Expression)
			\item c $\in$ commands 
			\item X $\in$ Loc    (Location).
		\end{itemize}
			
	\textbf{Formationsregeln}: abstrakte Syntax , indem neue Expressions erstellt werden .
    Hier interessieren wir nur die Bedeutung von Programm ,also nicht wie können das aufschreiben.
	\begin{itemize}
	
			\item a::= n $\mid$ $a_0$ + $a_1$ $\mid$ $a_0$ - $a_1$ $\mid$ $a_0$ * $a_1$.
			\item b::= true $\mid$ false $\mid$ $a_0$ == $a_1$ $\mid$ $a_0$ $\geq$ $a_1$ $\mid$ $a_0$ $\leq$ $a_1$.    
			\item c ::= skip $\mid$ X := a $\mid$  $c_0$, $c_1$ $\mid$ if b then $c_0$ else $c_1$ $\mid$ while b do c .	
		\end{itemize}
	Hinweise :
	\begin{itemize}
		\item wenn $a_0$ und $a_1$ arithmetische Ausdrücke sind, ist es auch die Resultat von $a_0$ - $a_1$ etc. 
		\item (\textbf{::=}) sollte gelesen werden als " kann sein " .
		\item (\textbf{$\mid$}) sollte gelesen werden als (oder) .
	\end{itemize} 
Beispiel : Die Boolean Expression b kann als True sein , wenn die Bedienung erfüllt ist , sonst False . 

\subsection{Syntaktische Menge }
Angenommen , dass wir zwei Elemente haben $a_0$ , $a_1$  $\in$ gleiche syntaktische Menge . wenn wir so  $a_0$ == $a_1$ schreiben, heißt das $a_0$ und $a_1$ gleich Werte haben oder identisch sind . 
Die  Arithmetic Expression 12 - 2  wird von ganze zahlen aufgebaut. aber sie sind nicht Syntaktisch identisch mit 10 , obwohl wir wissen , dass die Ergebnis identisch ist . so 12-2 ist nicht  Syntaktisch identisch mit 10.

\section{Auswertung}
Arithmetic Expression wird in Integer ausgewertet und Boolean Expression wird in True oder False value ausgewertet.
wie ein Programm verhaltet sich ?
\begin{itemize}
	\item<1-> Status definieren
	\item<2-> Auswertung der Boolean Expression und Arithmetic Expression.
	\item <3-> Kommando führt aus
\end{itemize}

\subsection{Auswertung von Arithmetic Expression}
Die Auswertung eines arithmetischen Ausdrucks a in einem Zustand  $\sigma$ to n.

\begin{center}  (a,$\sigma$) $\to$ n.    
\end{center}
                      
Auswertung eines Arithmetic Expression  ($a_0$ * $a_1$):
\begin{itemize}
\item wir werten  $a_0$  in $n_0$ aus
\item wir werten $a_1$ in $n_1$ aus
\item Jetzt können wir $n_0$ * $n_1$ und n als Ergebnis von $a_0$ * $a_1$ erhalten.
\end{itemize}
Also brauchen wir Regeln für die Auswertung.

\textbf{Regeln} :\begin{enumerate}

\item  Auswertung von Nummern : Alle Nummern sind schon ausgewertet.\label{1}
\begin{center}
(n,$\sigma$) $\to$ n.    

 ....no premises braucht   $\frac{ } {\text{(n,$\sigma$) $\to$ n}}$
\end{center}  
\item Auswertung von Location : wird in seiner Inhalt ausgewertet .\label{2}

\begin{center}
	(X,$\sigma$) $\to$ $\sigma(X)$.    
	
	....no premises braucht    $\frac{ } {\text{(X,$\sigma$) $\to$ $\sigma(X)$}}$
\end{center} 

\item \label{3} Auswertung von Sum :  n ist die Eregebnis von $n_0$ + $n_1$

\begin{center}
	 $\frac  {\text {($a_0$ , $\sigma$) $\to$ $n_0$   ($a_1$ , $\sigma$) $\to$ $n_1$} } { \text{($a_0$+$a_1$,$\sigma$) $\to$ n} }$ 
\end{center}

\item  \label{4} Auswertung von Subs:n ist die Eregebnis von $n_0$ - $n_1$

\begin{center}
	$\frac  {\text {($a_0$ , $\sigma$) $\to$ $n_0$   ($a_1$ , $\sigma$) $\to$ $n_1$} } { \text{($a_0$-$a_1$,$\sigma$) $\to$ n} }$
	
	\end{center}

\item  \label{5} Auswertung von Prod: n ist die Eregebnis von $n_0$ * $n_1$

\begin{center}
	$\frac  {\text {($a_0$ , $\sigma$) $\to$ $n_0$   ($a_1$ , $\sigma$) $\to$ $n_1$} } { \text{($a_0$*$a_1$,$\sigma$) $\to$ n} }$
	\end{center}

\end{enumerate}   
Hinweis: Regeln ohne premises : heißt Axiome. 



Beispiel : \text{((Init+30)-(7+9),$\sigma$)$\to$ 14 }
\begin{center} \fontsize{18}{15}
	
	$\frac{  \text{  $\frac { \text {$\frac  {}  {\text{ (Init,$\sigma$)$\to$0 $\ref{2}$ $\quad$          (30,$\sigma$)$\to$ 30 $\ref{1}$ }}$}   } {\text{(Init+30,$\sigma$)$\to$30  $\ref{3}$}}$ $\qquad$   $\frac { \text {$\frac  {}  {\text{ (7,$\sigma$)$\to$7 $ $\ref{1}$\quad$           (9,$\sigma$)$\to$9 $\ref{1}$}}$}} {\text{(7+9,$\sigma$)$\to$16 $\ref{3}$}}$  }} {\text{((Init+30)-(7+9),$\sigma$)$\to$ 14 $\ref{4}$}}$.
	
	
	
	*Hinweis:Init ist eine location mit $\sigma$(Init) = 0. 
\end{center}

Der Stil der operationelle  Semantik , die hier genutzt wird,ist struktural ( small-step semantics).
\begin{itemize}
\item struktural :Syntaxorientiert und induktiv.
\item Die Semantik eines Programms wird durch  die Semantik seiner Teile definiert.


Der Zustandsübergang für 7 + 9 wird mit dem Transition für 7 \ref{1} und dem Transition für 9 \ref {1} beschrieben.
\end{itemize}
              
\subsection{Auswertung von Boolean Expression}       
 \textbf{Regeln} :
 \begin{enumerate}
 \item \label{11} (true,$\sigma$)$\to$true
	\item \label{12} (false,$\sigma$)$\to$false
	\item \label{13}  $\frac  {\text {($a_0$ , $\sigma$) $\to$ $n_0$   ($a_1$ , $\sigma$) $\to$ $n_1$} } { \text{(($a_0$==$a_1$),$\sigma$) $\to$ true} }$  
	wenn $a_0$ gleich $a_1$ ist.
	\item \label{14}  $\frac  {\text {($a_0$ , $\sigma$) $\to$ $n_0$   ($a_1$ , $\sigma$) $\to$ $n_1$} } { \text{(($a_0$==$a_1$),$\sigma$) $\to$ false} }$
	wenn $a_0$ ungleich $a_1$ ist.
	\item \label{15}  $\frac  {\text {($a_0$ , $\sigma$) $\to$ $n_0$   ($a_1$ , $\sigma$) $\to$ $n_1$} } { \text{(($a_0$ $\geq$ $a_1$),$\sigma$) $\to$ true} }$
	wenn $a_0$ gleich oder grosser als $a_1$ ist.
	\item \label{16}  $\frac  {\text {($a_0$ , $\sigma$) $\to$ $n_0$   ($a_1$ , $\sigma$) $\to$ $n_1$} } { \text{(($a_0$ $\geq$ $a_1$),$\sigma$) $\to$ false} }$
	wenn $a_0$ ungleich oder kleiner als $a_1$ ist.
\end{enumerate}   
unsere ziel ist  auszuwerten die Boolean Expression in true oder false Value . 

\section{Ausführung von Kommandos}     
                     
\subsection{Theorem-Umgebung}

\begin{definition}[Gerade Zahlen]
 Eine ganze Zahl $z \in \mathbb Z$ heißt \emph{gerade}, wenn sie durch 2 teilbar ist.
\end{definition}

\begin{theorem}[Optionaler Titel]
 Es gibt voll viele gerade Zahlen.
\end{theorem}
\begin{proof}
 2 ist zum Beispiel eine. Oder 18, 0, -100 oder 42.
\end{proof}


%------------------------------------------------
% Literaturverzeichnis
%------------------------------------------------

\bibliographystyle{abbrvnat-ger}
\bibliography{literatur}
\addcontentsline{toc}{section}{\bibname}

\end{document}
